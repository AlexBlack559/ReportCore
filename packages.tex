\usepackage{mathtools}               % Для математики и формул (asmath+плюшки)
\usepackage{amsthm}                  % Для теорем и доказательств
\usepackage{fullpage}                % Чтобы контент на всю страницу
\usepackage{todonotes}               % TODO
\usepackage{tikz}                    % Для рисунков и графиков
\usepackage{pgfplots}                % Графики
\usepackage[warn]{mathtext}          % Для матем. символов
\usepackage[T2A]{fontenc}            % Для русского шрифта
\usepackage[utf8]{inputenc}          % Для UTF-8 в исходниках
\usepackage{dsfont}                  % Для букв множеств чисел
\usepackage{amssymb}                 % Для символа неравно и др. мат. символов
\usepackage[english, russian]{babel} % Для русского в структуре документа
\usepackage{mdframed}                % Теоремы в рамочках
\usepackage{geometry}                % Гибкие настройки размеров страницы
\usepackage{etoolbox}                % Для условных операторов
\usepackage{cancel}                  % Для зачеркивания формул
\usepackage{ifthen}                  % Для условных операторов
\usepackage{chngcntr}                % Для греческого Ро
\usepackage{enumerate}               % Для кастомное энумерации
